% !Mode:: "TeX:UTF-8"

\hitsetup{
  %******************************
  % 注意:
  %   1. 配置里面不要出现空行
  %   2. 不需要的配置信息可以删除
  %******************************
  %
  %=====
  % 秘级
  %=====
  statesecrets={公开},
  natclassifiedindex={TM301.2},
  intclassifiedindex={62-5},
  %
  %=========
  % 中文信息
  %=========
  ctitleone={基于深度强化学习的},%本科生封面使用
  ctitletwo={异构系统频谱智能分配},%本科生封面使用
  ctitlecover={基于深度强化学习的异构系统\\频谱智能分配},%放在封面中使用,自由断行
  ctitle={基于深度强化学习的异构系统频谱智能分配},%放在原创性声明中使用
  %csubtitle={一条副标题}, %一般情况没有,可以注释掉
  cxueke={工学},
  csubject={通信工程},
  caffil={电子与信息工程学院},
  cauthor={陈鹏},
  csupervisor={高玉龙},
  %  cassosupervisor={某某某教授}, % 副指导老师
  %  ccosupervisor={某某某教授}, % 联合指导老师
  % 日期自动使用当前时间,若需指定按如下方式修改:
  cdate={2020年6月11日},
  cstudentid={1160500920},
  %cstudenttype={同等学力人员}, %非全日制教育申请学位者
  %(同等学力人员)、(工程硕士)、(工商管理硕士)、
  %(高级管理人员工商管理硕士)、(公共管理硕士)、(中职教师)、(高校教师)等
  %
  %
  %=========
  % 英文信息
  %=========
  etitle={Research on key technologies of partial porous externally pressurized gas bearing},
  esubtitle={This is the sub title},
  exueke={Engineering},
  esubject={Computer Science and Technology},
  eaffil={\emultiline[t]{School of Mechatronics Engineering \\ Mechatronics Engineering}},
  eauthor={Yu Dongmei},
  esupervisor={Professor XXX},
  eassosupervisor={XXX},
  % 日期自动生成,若需指定按如下方式修改:
  edate={December, 2017},
  estudenttype={Master of Art},
  %
  % 关键词用“英文逗号”分割
  ckeywords={分簇模型, 频谱分配, 深度强化学习, 长短期记忆网络},
  ekeywords={cluster model, spectrum allocation, DRL, LSTM},
}

\begin{cabstract}

随着未来空天地一体化通信框架的提出,各种系统的用户将共享频谱资源。这使得本就紧张的频谱资源变得更为珍贵,为此本文考虑异构系统中频谱的动态与智能分配问题,将其构建其为部分观测马尔可夫决策过程。在空天地一体化通信框架下,本文提出以用户为中心的分布式智能资源分配,提出分簇模型代替原有地面蜂窝网架构,让用户以自组织形式成为簇头或簇内用户并进行资源自主分配。簇头在考虑用户数量及用户移动性等因素后自主决策簇大小及资源提供总量,用户对这些资源进行申请,最终由簇头进行分配。将通信链路建立和资源分配流程细分为三个部分:簇头选择,信道分配,功率控制。在传统算法难以解决的前提下,使用深度强化学习解决该问题。以集中离线训练方式对当前可接入簇头状态进行预测,给出最优簇头选择方案,方案中用户通过相同网络构架DRQN选择信道,避免冲突,最后在簇头用户成功连接前提下进行功率方面调节。在保证用户通信体验同时提高频谱的利用效率,使整个网络吞吐量最大化。

本文在传统的强化学习(RL)的基础上融合深度神经网络,形成深度强化学习(DRL),提高智能体对输入数据为连续信息情况下的识别能力。加入长短期记忆网络(LSTM),提高智能体对连续历史信息的处理能力,使其可以依据既往信息对当前状态进行有效预测。通过强化学习独有的决策能力获取最优动作,以此得到问题近似最优解。通过借鉴已有方法将网络切分为状态网络和动作优势网络,并加入目标网络形成双网络架构来优化DRQN网络,使智能体在实现高效训练,快速收敛的基础上获得更稳定的收敛效果。

在具体仿真和验证过程中给出具体操作流程,依据问题的不同细节进行算法调整,并对仿真测试结果进行直观化展示以及分析,验证算法适用性以及对比传统算法优势,得到较为满意结果。初步形成一个完整,可行,有效的新型通信框架下的异构系统资源分配算法。并于最后对已完成工作进行总结评价,并为未来频谱资源高效利用提出创新性建议。

\end{cabstract}

\begin{eabstract}
   
   With the proposed future integrated communication framework for space, air and ground, users of various systems will share the scarce spectrum resources.We consider the problem of spectrum allocation in a dynamic heterogeneous system which is formulated as a partially observable Markov decision process. Under the frame of Space-air-ground integration network(SAGIN), we propose a new model of cluster to replace the cellular network which is user-centric distributed intelligent resource allocation. Users can self-organized to be cluster head or in-cluster users and they can autonomous allocation of resources.The communication link establishment and resource allocation process is subdivided into three parts: cluster head selection, channel allocation, and power control. We apply deep reinforcement learning to solve the problem. We train the agent at central unit in an offline manner to have the ability of predicting the current state of the cluster head. In addition, the agent can use the same network frame to choose a specific channel and get on power control. Therefore the spectrum utilization can be improved and maximize whole network throughput.
   
   We merge deep neural network(DL) with the reinforcement learning(RL), which come to deep reinforcement learning(DRL),it can settle high-dimensional input data. In addition, we add LSTM to predict the state according to the past information. By referring to existing methods, the network is divided into state network and action advantage network, and the target network is added to form a double network architecture to optimize the DRQN network, so that the agent can obtain more stable convergence effects on the basis of efficient training and rapid convergence. 

   In the specific simulation and verification process, the algorithm is adjusted and improved according to the different details of the problem, and the simulation test results are visually displayed and performance analysis. result. A complete, feasible, and effective new communication resource allocation algorithm for heterogeneous systems is initially formed. And at the end, it summarizes and evaluates the completed work, and puts forward innovative suggestions for the efficient use of spectrum resources in the future.
\end{eabstract}
