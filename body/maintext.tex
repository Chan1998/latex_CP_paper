\chapter{空天地一体化}
\section{引言}
空天地一体化通信框架融合卫星通信,空基无人机(UAV),地面移动通信为一体,三者优劣互补,可为未来通信提供更广覆盖,更低延时,更大带宽,更高可靠的新一代通信服务。对于传统地面蜂窝网结构进行改进,提出基于分簇模型的新型框架,使用户在作为服务中心的同时成为资源分配中心,可以更自主,更便捷,也更高效合理的实现资源分配。本章将对空天地一体化模型进行展示,并着重介绍我们提出的分簇模型框架,并将对设计的相关问题的数学模型进行叙述。

\section{空天地一体化新型通信框架模型的建立}
对于现有传统通信系统如图\ref{fig:现有框架}所示。主要包括卫星通信系统,空基无人平台以及地面通信网络。其中地面段通信系统较为完善,而对于包含高空平流层的空天地一体化的通信系统模型研究还并不完善,这里我们需要对于异构系统进行简化。
\begin{figure}[h]
	\centering
	\includegraphics[width = 0.8\textwidth]{一体化框架}
	\caption{现实情况下异构系统通信框图}
	\label{fig:现有框架}
\end{figure}

简化后的模型如图\ref{fig:简化模型}所示。卫星可以直接与地面终端进行通信,同时也可以经由空中无人平台进行中转和广播,由于设计协议复杂,我们现就问题主要矛盾,即地面段异构系统的频谱分配进行研究,在算法较为成熟时再考虑空间段的具体细节。

\begin{figure}[h]
	\centering
	\includegraphics[width = 0.7\textwidth]{简化模型}
	\caption{简化后的新型通信框架}
	\label{fig:简化模型}
\end{figure}

\section{分簇模型}
\subsection{分簇模型结构}
这里不同与传统的蜂窝网概念,我们创新性的提出用户为中心自组织式的分簇模型。在我们构建的新型通信模型下,作为中介节点的不止有基站,可能还有一些设备能力较强的终端用户形成自组织式的小小区,我们将之称为簇头,现在主要讨论簇内与簇间在考虑同频干扰情况下的频谱分配算法的优化。模型构建类似于地面移动通信的蜂窝网模型,但与之不同点在于基站的角色由簇头承担,簇头可以是现存基站,也可以是具有较强计算能力的移动终端,分簇模型如图\ref{fig:分簇模型结构图}所示。这里我们假设每个簇头可能连接多个簇内用户为他们提供信息传输服务,每个用户也可能同时连接多个簇头,通过训练后的分布式智能体( 簇内用户)可以自主选择连接哪些簇头,然后在被告知共用用户数量后再自主选择特定信道申请分配。这些都结束后,中心智能体(簇头)会根据申请连接情况进行智能功率控制,以此达到网络总传输速率最高。
\begin{figure}[h]
	\centering
	\includegraphics[width = 1.1\textwidth]{分簇模型结构图}
	\caption{简化后的新型通信框架}
	\label{fig:分簇模型结构图}
\end{figure}

\subsection{分簇模型建立原因及意义}
关于分簇原因我们这里列出如下:

(1) 动态的网络:用户和基站都将具有相当的移动性。用户的移动性不再赘述,基站的移动性主要是指一体化组网构架下,为了弥补地面网络带宽受限的问题而引入了平流层的飞艇作为移动的基站。在这种状况下,不得不通过分簇来解决动态性的问题。

(2) 分布式的系统:未来系统可能是分布式的,之前的工作也已经说明,分布式系统中的用户联系松散,需要根据特定的情况来保持和特定用户群体的联系。因此,分簇和分布式的系统也是密切相关的。

(3) 用户需求多样化的网络:用户对通信的需求大相径庭,需要打破小区不重叠、 在哪个小区用哪个基站的限制,让用户自由选择节点、甚至组成以自己为中心的簇,自主地调配资源,为自己服务。

\section{问题的数学模型}
这里我们对相关问题进行简化和抽象,并进行数学建模,为后续问题解决提供体系化的理论支持。首先我们给出整个问题的完整流程图如\ref{fig:整体流程图}所示。
\begin{figure}[htbp]
	\centering
	\includegraphics[width = 0.9\textwidth]{分簇频谱分配算法}
	\caption{异构系统频谱智能分配整体流程图}
	\label{fig:整体流程图}
\end{figure}

待解决问题主要分为三个部分,实现簇头自主选择部分,我们简化问题为基于历史信息的相关信道状态预测问题,定义为部分观测的马尔可夫决策过程(POMDP)。然后是实现分布式多用户冲突避免问题,我们建模为纳什均衡问题,也称非合作博弈均衡,在仅知道竞争用户数量情况下通过训练使智能体达到纳什均衡状态。最后是功率控制部分,我们建模为认知无线电下次级用户的功率退避问题。

对于簇头的选择是用户依据历史信息对不同簇头当前转发状态与能力的估计,在之前的时隙中(timeslot),用户在每个时间隙发出信息传输请求,得到信息传输结果观测值(ACK),以此估计当前簇头转发状态好坏,输出一个布尔值(0/1),这里假设不同簇头之间是相关的,由于不能完全知晓当前簇头状态转移模式,因此是一个部分观测马尔可夫决策过程,如表格\ref{tab:马尔可夫分类}所示。对于挑选簇头问题可以建模为一个$2^{N}$状态的马尔科夫链,\textit{N}表示用户可连接簇头数目,状态空间为$S=\left\{S_{1},\cdots,S_{2^{N}}\right\}$,其中每个$S_{i}$,为一个长度为\textit{N}的向量,表示每个簇头当前状态。用户通过历史信息进行预测,并更新自身的系统状态分布推测$\Omega \left ( t \right )=\left [ \omega _{S_{1}} \left ( t \right ),\cdots,\omega _{S_{2^{N}}} \left ( t \right )\right ]$,每个向量元素表示系统位于当前状态概率,并依据此估计给出最优动作选择向量$a(t)$并获取ACK,以此根据公式\ref{eq:簇头选择更新公式}(式中$\mathbb{I}\left(\cdot\right)$为指示函数)更新系统状态猜测向量\cite{8303773}。
\begin{equation}\label{eq:簇头选择更新公式}
\hat{\omega }_{S_{i}}\left ( t \right )=\left\{
\begin{aligned}
\frac{\omega _{s_{i}\left ( t \right )}\mathbb{I}\left ( S_{ik}\left ( t \right ) =1\right )}{\sum_{i=1}^{2^{N}}\omega _{s_{i}\left ( t \right )}\mathbb{I}\left ( S_{ik}\left ( t \right ) =1\right )}& & a(t)=k,o(t)=1\\
\frac{\omega _{s_{i}\left ( t \right )}\mathbb{I}\left ( S_{ik}\left ( t \right ) =1\right )}{\sum_{i=0}^{2^{N}}\omega _{s_{i}\left ( t \right )}\mathbb{I}\left ( S_{ik}\left ( t \right ) =0\right )}& & a(t)=k,o(t)=0
\end{aligned}
\right.
\end{equation}
然后将猜测向量与每个簇头不同概率分布组成的状态转移矩阵\textit{P}做乘积得到下一状态概率矩阵$\Omega \left ( t+1 \right )$,以此达到长期累积折扣奖赏最大的优化目标,获得最优策略,如公式\ref{eq:簇头选择长期奖励}所示。其中$R_{\pi\left ( \Omega \left ( t \right ) \right )}\left ( t \right )$为t时刻在当前状态$\Omega \left ( t \right ) $下,最佳策略$\pi^{*}$采取动作后得到的奖赏。折扣因子$\gamma$为介于0和1之间的数,用来表示长期奖励对总任务的重要程度。
\begin{equation}\label{eq:簇头选择长期奖励}
\pi ^{*}= \underset{\pi }{\arg \max}\mathbb{E}_{\pi}\left [ \sum_{t=1}^{\infty }\gamma ^{t-1} R_{\pi\left ( \Omega \left ( t \right ) \right )}\left ( t \right )\mid \Omega \left ( 1 \right )\right ]
\end{equation}
当簇头间相互独立且簇头状态转移概率分布形式一致时,每个用户基于自身的贪心算法即为最优策略\cite{Ahmad2009Optimality},但对于簇头间相互关联且变化概率分布未知的情况下这种算法得到的结果并不是最优化的,且由于簇头间相关性,该问题为一个PSPACE-hard问题\cite{8303773},很难通过传统算法得到精确解。
\begin{table}[htbp]
	\caption{马尔可夫模型类别}\label{tab:马尔可夫分类}
	\vspace{0.5em}\centering\wuhao
	\begin{tabular}{cccc}
		\toprule[1.5pt]
		名称 & 状态可见性 & 是否考虑动作 & 英文缩写 \\
		\midrule[1pt]
		马尔科夫链 & 状态完全可见 & 不考虑动作影响 & MC \\
		马尔可夫决策过程 & 状态完全可见 & 考虑动作影响 & MDP\\
		隐马尔可夫模型 & 状态不完全可见 & 不考虑动作影响 & HMM \\
		部分观测马尔可夫决策过程 & 状态不完全可见 & 考虑动作影响 & POMDP \\
		\bottomrule[1.5pt]
	\end{tabular}
\end{table}


对于分布式多用户对有限信道的竞争问题建模为纳什均衡问题。这里给出纳什均衡的定义在博弈$ G=\left \{ S_{1},\cdots,S_{n}:U_{1},\cdots,U_{n}\right \}  $中,每个博弈方采取一个策略,组成一个策略集合$\left\{S_{1}^{*},\cdots,S_{n}^{*}\right\}$,对于其中任意一个博弈成员$i$的策略$S_{i}^{*}$都是对其余博弈者采取策略组合$\left\{S_{1}^{*},\cdots,S_{i-1}^{*},S_{i+1}^{*},\cdots,S_{n}^{*}\right\}$的最佳应对策略,也即$U_{i}\left \{ S_{1}^{*}, \cdots ,S_{i}^{*},\cdots,S_{n}^{*} \right \}\geqslant U_{i}\left \{ S_{1}^{*},\cdots ,S_{ij}^{*},\cdots,S_{n}^{*} \right \}$对于任意$S_{ij}\in S_{i}$恒成立,我们称对应的策略集合为$G$的一个纳什均衡。通俗来讲就是其他竞争对手不改变自身竞争策略时候,自己不改变当前策略才会获取利益最大值,也就是达到平衡状态。对于信道竞争问题来说,即为在分布式用户间不进行合作式信息交流前提下达到平衡状态的问题。这里我们使用$\mathcal{N}=\left \{ 1,2,\cdots,N \right \}$表示簇内竞争用户然后使用集合$\mathcal{K}=\left \{0,1,2,\cdots,K \right \}$表示选择的信道(如果选择动作为0则表示用户在此时隙选择避让,不进行信息传递),在每个时隙,同一簇内不同用户选择一个簇头可提供信道发出信息传输请求,如果和其他用户不发生冲突则返回成功传输信号(ACK),反之亦然。目标仍然是达到长期总传输成功率最高。这里我们仍然使用带有折扣系数$\gamma$的累积奖赏作为优化目标,如式\ref{eq:冲突避免奖赏公式}所示。
\begin{equation}\label{eq:冲突避免奖赏公式}
R_{n}=\sum_{i=1}^{N}\sum_{t=1}^{\infty }\gamma ^{t-1}r_{n}\left ( t \right )
\end{equation}
由于用户间不能相互通信,所以普通算法对于这种问题只能使用随机信道选取方式,导致信道出现空闲或冲突,致使频谱利用效率降低,本文将使用深度强化学习解决这一问题。

对于多用户功率功率控制问题,我们将其设定为在认知无线电框架下的次级用户的功率退避问题。在完成簇头选择和信道选择后,簇头和簇内用户成功建立通信链路,但由于多用户的存在以及用户移动变化等因素,如果不采取相应的动态功率控制肯能导致同频干扰或是邻频干扰对信息传递造成巨大干扰,降低通信服务质量。简化起见,我们考虑簇内有个用户,并将其中一个假设为主用户(PU, primary user),另一个为次级用户(SU, secondary user),且每个用户都拥有自己的发射机($T_{X_{1}},T_{X_{2}}$)与接收机($R_{X_{1}},R_{X_{2}}$),两者属于非合作型共存方式,主用户不需要考虑次级用户,并且仅依据自己固定的原有功率策略进行功率调整。次级用户不知道主用户通讯功率及其功率策略,需要通过感知周围环境对主用户功率变化产生响应,及时调整自己的功率参数,以达到共同通信目的。这里需要特别提出的是,虽然主用户与次级用户不产生直接信息交流,但次级用户功率调整会通过环境变化引起主用户做出功率改变,也即客观上两者会产生相互作用。方便起见,我们假设两者同步进行功率调整,拥有相同时间基线。同时有感应节点部署在环境中获取接收信号强度(RSS, received signal stength),协助次级用户估计主用户传输状态。我们使用信干噪比SINR表示各用户QoS,如公式\ref{eq:SINR}所示,式中$h_{ij}$表示对应信道增益,$N_{i}$为对应噪声功率,主用户功率策略\cite{Grandhi1998Constrained}为如下公式所示\ref{eq:传统功率策略}。式中$D\left ( \cdot  \right )$为离散式功率选择函数,选取离散功率集合以$\mathcal{P}_{1}\triangleq\left \{ p_{1}^{p},\cdots,p_{L_{1}}^{p} \right \}$表示。$SINR_{1}(k)$为主用户测量得到的信干噪比,$p_{1}(k)$为主用户选择的功率。环境状态使用$p_{n}^{r}(k)=p_{1}(k)g_{1n}+p_{2}(K)g_{2n}+w_{n}(k)$进行描述,$p_{1}(k)g_{1n},\;p_{2}(K)g_{2n}$分别表示两用户发射功率同各自增益乘积,$w_{n}(k)$表示零均值高斯白噪声。
\begin{equation}\label{eq:SINR}
SINR_{i}=\frac{\left \| h_{ii} \right \|^{2}p_{i}}{\sum_{j\neq i}\left \| h_{ji} \right \|^2p_{j}+N_{i}}\; i=1,2 
\end{equation}
\begin{equation}\label{eq:传统功率策略}
p_{1}\left ( k+1 \right )=D\left ( \frac{\eta _{1}p_{1}\left ( k \right )}{SINR_{1}\left ( k \right )} \right )
\end{equation}
随后次级用户在智能体决策下选择一个功率集合$\mathcal{P}_{2}\triangleq\left \{ p_{1}^{s},\cdots,p_{L_{2}}^{s} \right \}$中的功率进行信息传输。当两用户都满足QoS要求($SINR_{i}\geqslant\eta_{i}$)时,此时刻返回奖赏为1,最终目标为整体长时累积奖惩最大化。
\section{本章小结}
在这一章节中,我们对整个问题的背景框架以及整体算法流程进行了介绍,并提出一种基于分簇理论的新型通信框架,在此基础上对于问题涉及的簇头选择,信道分配,功率控制这三方面进行了具体的数学语言描述,分别建模为部分观测马尔可夫决策过程,纳什均衡问题以及认知无线电框架下次级用户的功率退避问题,形成一个完备严密的逻辑体系,为后续工作奠定理论基础。



\chapter{深度强化学习算法}
\section{引言}
上一章对问题背景框架及数学模型进行了阐述,本章将会对我们使用的机器学习算法原理进行论述包含强化学习中的\textit{Q}表学习和融入深度神经网络的DQN算法,并在前人基础上加入新的想法,将长短时记忆网络(LSTM)加入DQN中,并对网络结构进行相应优化,比如增加训练效率的duling方法和为了网络训练更稳定而加入的目标\textit{Q}网络。
\section{深度强化学习相关理论}
\subsection{\textit{Q}学习}
首先介绍强化学习,作为机器学习的一种,强化学习主要包括智能体(Agent),环境(Environment),状态(State),动作(Action),奖惩值(Reward)组成,如图\ref{fig:强化学习}所示。将其如认知无线电框架\ref{fig:认知无线电}相类比,可见结构较为相似,这也是我们认为可以使用强化学习方法解决频谱分配问题的原因之一。
\begin{figure}[htbp]
	\begin{minipage}{\textwidth}
		\centering
		\subfigure{\label{fig:强化学习}}\addtocounter{subfigure}{-2}
		\subfigure{\subfigure[强化学习框架]{\includegraphics[width=0.25\textheight]{强化学习}}}
		\hspace{1em}
		\subfigure{\label{fig:认知无线电}}\addtocounter{subfigure}{-2}
		\subfigure{\subfigure[认知无线电框架]{\includegraphics[width=0.3\textheight]{认知无线电框架}}}
		\hspace{1em}	
	\end{minipage}
	\vspace{0.2em}
	\caption{强化学习与认知无线电类比}\label{fig:强化学习类比认知无线电}
\end{figure}
整个过程是智能体与环境不断交互中进行的,在环境中,智能体执行一个动作后,与环境进行交互,环境依据选取的动作给出反馈(积极的或是消极的),智能体会依据反馈进行自我训练(例如\textit{Q}表更新或是神经网络权值改变等),随后,智能体会根据环境的反馈和新的状态按照一定的策略执行新的动作。如此循环往复,直至任务结束。强化学习目标是累积奖赏值的最大化,如式\ref{eq:强化学习目标}所示。强化学习的常见模型是标准的马尔可夫
\begin{equation}\label{eq:强化学习目标}
G_{t}=R_{t+1}+\lambda R_{t+2}+\cdots=\sum_{k=0}^{\infty}\lambda^{k}R_{t+k+1}
\end{equation}
决策过程,侧重于训练学习并在未遍历选择间的探索和已知最优化选择之间的平衡。与监督学习和非监督学习不同的是,强化学习不需要提前准备大量的训练数据,其训练数据是通过接收环境对动作的反馈而来,并由此进行模型更新。

\begin{figure}[h]
	\centering
	\includegraphics[width = 1.0\textwidth]{Q学习}
	\caption{\textit{Q}学习示意图}
	\label{fig:Q学习}
\end{figure}

强化学习中最为典型的就是\textit{Q}表学习算法,其算法结构如图\ref{fig:Q学习}所示。其中\textit{Q}-learning算法是\textit{Q}学习的核心算法。算法最终目标为找到一个使累积折扣奖赏最大的策略,如式\ref{eq:Q学习目标}所示。
\begin{equation}\label{eq:Q学习目标}
\underset{\pi}{\max }\mathbb{E}\left [ \sum_{t=0}^{H}\gamma^{t}R\left ( S_{t},A_{t},S_{t+1} \right )\mid\pi \right ]
\end{equation}
算法使用时间差分法(TD算法),同时融合蒙特卡洛(深度优先)和动态规划(广度优先)方法,能够进行离线学习。利用贝尔曼方程\ref{eq:贝尔曼方程}可以对符合马尔可夫决策过程的问题求解最优策略。每个状态的价值不仅取决于当前状态,还需要考虑后续状态的预估价值。
\begin{equation}\label{eq:贝尔曼方程}
\left\lbrace 
	\begin{aligned}
	V_{\pi(s)}=&\mathbb{E}\left ( U_{t}\mid S_{t=s} \right )\\
	V_{\pi(s)}=&\mathbb{E}_{\pi}\left [ R_{t+1}+\gamma \left [ \cdots \right ]\mid S_{t=s} \right ]\\
	V_{\pi(s)}=&\mathbb{E}_{\pi}\left [ R_{t+1}+\gamma V\left ( {s}' \right )\mid S_{t=s} \right ]
	\end{aligned}
	\right.
\end{equation}
绝大多数情况下我们使用价值函数$V_{\pi(s)}$表征某一策略下某个状态的价值,并使用动作价值函数$Q(s,a)$表征某一状态动作对对于整个问题的价值大小,也即在某一状态下采取某一动作的好坏,因此,\textit{Q}学习又被分类为基于价值的强化学习。最优状态动作函数如式\ref{eq:Q函数}所示。
\begin{equation}\label{eq:Q函数}
Q^{*}\left ( s,a \right )=\sum _{{s}'}P\left ( {s}'\mid s,a \right )\left ( R\left ( s,a,{s}' \right ) +\gamma\max _{{a}'}Q^{*}\left ( {s}' ,{a}'\right )\right )
\end{equation}
对公式直观的解释是当前的动作价值函数等于将执行完该动作之后所有可能达到的下一状态进行遍历,并借助贝尔曼方程将下一动作置换为对应奖赏和执行当前策略的最优动作时的动作价值函数的加和形式。以此引入当前策略下的下一状态的动作价值函数,形成迭代关系,通过训练使迭代趋于收敛。为了完成这样的迭代训练,我们需要引入\textit{Q}算法的学习机制,由公式\ref{eq:Q学习更新函数}给出。
\begin{equation}\label{eq:Q学习更新函数}
\left\lbrace 
	\begin{aligned}
	&Q_{k+1}\left ( s_{t},a_{t} \right )=Q_{k}\left ( s_{t},a_{t} \right )+\alpha_{k}E_{k}\\
	&E_{k}=r_{t}+\gamma\max Q_{k}\left ( {s_{t}}' ,{a_{t}}'\right )-Q_{k}\left ( s_{t},a_{t}\right )
	\end{aligned}
\right.
\end{equation}
当训练至损失函数$E_{k}$小于提前设定的一个较小的阈值$\epsilon$时认为迭代收敛,训练停止,此时\textit{Q}表成功建立,任意输入某一状态$s(t)$都可以由\textit{Q}表查询到对应的最优动作,也即完成了最优策略的寻找。
\subsection{深度\textit{Q}学习}
传统的\textit{Q}表学习对于拥有维度较小的状态空间和动作输出空间的问题有较好的解决效果,但其有一个瓶颈就是因为要使用一个表格储存所有的状态动作函数,当一个复杂问题拥有较高的输入状态维度和较大的动作输出维度时,\textit{Q}表学习就显得有些力不从心。一方面由于连续状态存在,表格索引这种离散化的描述方式使得智能体对于动作价值函数的描述变得精度不足。另一方面,大量的状态动作维度使得表格变得十分巨大,在训练时更新带来灾难性的计算复杂度和时间复杂度,在测试时对表格的遍历也变得十分困难。这些问题极大程度的限制了传统\textit{Q}学习在频谱分配相关问题上的应用。
\begin{figure}[h]
	\centering
	\includegraphics[width = 0.7\textwidth]{DQN}
	\caption{DQN算法示意图}
	\label{fig:DQN}
\end{figure}

近年来深度学习发展迅速,成为机器学习中最耀眼的成果,深度神经网络对于复杂状态的感知能力极为强大。有研究学者将深度神经网路的感知能力同强化学习的决策能力相结合,提出了深度强化学习\cite{mnih2013playing}概念,其中以DQN算法最为典型。DQN算法同传统\textit{Q}表学习算法最显著也是最核心的不同即为使用深度神经网络替代原有的\textit{Q}表格,使用深度神经网络对\textit{Q}函数进行近似,如图\ref{fig:DQN}所示。DQN以值网络作为评价模块,基于值网络的输出进行动作的选择。这里我们注意到DQN算法还加入了一个记忆单元,这个改变是由于深度神经网络训练的特殊性引起的。我们加入记忆单元储存每次交互产生的状态,动作,奖赏,下一状态组成的元组,并在训练时进行随机抽取,这样做的目的有两个,最主要的是打破输入数据间的相关性,降低网络陷入局部最优的概率,同时可以对数据进行自定义形式抽取,提高网络训练效率。这里对于误差函数的设置也较为特殊,类比监督学习,构建一个残差模型,我们将公式\ref{eq:Q学习更新函数}中的误差函数$E_{k}$作为训练优化目标,使用优化器利用梯度下降算法进行网络权值更新,随后进入下一步骤,多次与环境交互并训练,直至网络收敛,损失函数下降到设定阈值$\epsilon$之下,训练结束。

此外我们需要额外加入一个带有探索性质的动作选择策略,倘若在智能体通过环境交互过程中,智能体只选择当前网络给出的最优动作,有可能会使网络陷入局部最优的策略中,因为网络没有对未知动作进行探索,会导致记忆体内存储的历史经验不能完全包含相应问题的所有动作空间,这样的结果是我们所不愿见到的。为此,要利用$\epsilon-$贪心算法进行动作选择。算法核心是设置一个随训练轮次增加而逐步减小的阈值$\epsilon$,然后每次需要进行动作选取时生成一个随机数$p$,如果$p\leqslant \epsilon$此次动作选取使用随机探索方式选取,如果$p> \epsilon$选择网络输出的$\arg \max \left ( Q \right )$对应的最佳动作。

\subsection{长短期记忆网络}
科学家发现大脑皮层的解剖结构现实电流刺激信号可以在神经回路中循环存在,计算机科学家进行类比后提出反向回路设想。后来又有学者补充了循环神经网络的时间反向传播(BP Through Time, BPTT)算法,成为循环神经网络主要的学习方式。循环神经网络结构如图\ref{fig:循环神经网络}所示,其数学表达式如式\ref{eq:循环神经网络结构}所示。
\begin{figure}[h]
	\centering
	\includegraphics[width = 0.7\textwidth]{循环神经网络}
	\caption{循环神经网络结构}
	\label{fig:循环神经网络}
\end{figure}
\begin{equation}\label{eq:循环神经网络结构}
\left\lbrace 
\begin{aligned}
&O_{t}=g\left ( Vh_{t} \right )\\
&h_{t}=f\left ( Ux_{t} + Wh_{t-1} \right )
\end{aligned}
\right.
\end{equation}
其中$W$为网络各层连接的权值矩阵。$o_{t}$为相应时刻输出,$x_{t}$为相应时刻输入,$h_{t}$为神经节点状态值。网络采用基于时间的反向传播算法(BPTT)进行权值更新,这里数学推导颇为复杂,不作详谈。这种传统循环神经网络有一个缺点式由于长期依赖因素可能会导致梯度爆炸或消失,这时我们就需要其升级版本,长短期记忆网络(LSTM)。这里我们给出LSTM的网络结构示意图\ref{fig:LSTM示意图}。所有神经节点具有相同的结构,其核心是神经元状态,图中使用$C_{t}$表示,它贯穿整个神经元。
\begin{figure}[h]
	\centering
	\includegraphics[width = 1\textwidth]{LSTM}
	\caption{LSTM结构示意图}
	\label{fig:LSTM示意图}
\end{figure}
另外标注有$\sigma$矩形为神经元的遗忘门,是一个sigmoid函数同信息进行点乘,用来控制历史信息流入量。信息先经过第一个遗忘门,表示由神经元决定抛弃哪些信息如式\ref{eq:抛弃函数}所示。
\begin{equation}\label{eq:抛弃函数}
f_{t}=\sigma\left ( W_{f} \cdot \left [ h_{t-1},x_{t} \right ]+b_{f}\right )
\end{equation}
然后是一个输入门,决定为神经元状态新添加哪些信息,具体操如公式\ref{eq:LSTM输入公式}所示。使得当前神经元状态变为$C_{t}=f_{t}*C_{t-1}+i_{t}*\tilde{C}_{t}$。
\begin{equation}\label{eq:LSTM输入公式}
\left\lbrace 
\begin{aligned}
&i_{t}=\sigma \left ( W_{i} \cdot \left [ h_{t-1},x_{t} \right ] +b_{i}\right )\\
&\tilde{C}_{t}=tanh\left ( W_{C} \cdot \left [ h_{t-1},x_{t} \right ] +b_{C}\right )
\end{aligned}
\right.
\end{equation}
最后通过tanh函数进行激活,以此作为神经元的输出,如式\ref{eq:LSTM输出公式}所示。
\begin{equation}\label{eq:LSTM输出公式}
\left\lbrace 
\begin{aligned}
&o_{t}=\sigma \left ( W_{o} \cdot \left [ h_{t-1},x_{t} \right ] +b_{o}\right )\\
&h_{t}=o_{t}*tanh\left ( C_{t} \right )
\end{aligned}
\right.
\end{equation}
然后便是将LSTM加入传统深度神经网络之中,与DQN一同形成DRQN算法。这种深度强化学习因为加入了LSTM网络使得智能体可以通过历史信息对当前状态进行预测,可以更加有效的完成频谱智能分配任务。

\section{DRQN算法优化}
\subsection{Duling DQN}

\subsection{目标\textit{Q}网络}

\section{本章小结}




\chapter{异构系统频谱智能分配}
\section{引言}

\section{信道估计}
\subsection{基于真实数据的DRQN信道估计}

\subsection{信道估计结果展示及分析}

\section{冲突避免}
\subsection{分布式DRQN冲突避免}

\subsection{冲突避免结果展示及分析}

\section{功率控制}
\subsection{DRQN智能化功率控制}

\subsection{功率控制结果展示及分析}



\section{本章小结}






















\chapter{图片}


\section{引言}

\section{普通图片}

\subsection{示例}

\subsection{深度强化学习}
\begin{figure}[h]
	\centering
	\includegraphics[width = 0.8\textwidth]{频道选择,损失,tensorboard}
	\caption{频道选择,损失,tensorboard}
	\label{fig:channel choose loss tensorboard}
\end{figure}

\subsection{深度强化学习}
\begin{figure}[h]
	\centering
	\includegraphics[width = 0.8\textwidth]{动作概率,条形图}
	\caption{动作概率,条形图}
\end{figure}

这里引用了一个图片\ref{fig:vaeinput}.
这里再次引用了一个图片\ref{fig:channel choose loss tensorboard}.

\begin{figure}[htbp]
	\begin{minipage}{\textwidth}
		\centering
		\subfigure{\label{fig:vaeinput}}\addtocounter{subfigure}{-2}
		\subfigure{\subfigure[输入的图片]{\includegraphics[width=0.2\textheight]{频道选择,损失,tensorboard}}}
		\hspace{1em}
		\subfigure{\label{fig:vaeinput2}}\addtocounter{subfigure}{-2}
		\subfigure{\subfigure[输入的图片2]{\includegraphics[width=0.2\textheight]{频道选择,损失,tensorboard}}}
		\hspace{1em}
		\subfigure{\label{fig:vaeinput3}}\addtocounter{subfigure}{-2}
		\subfigure{\subfigure[输入的图片3]{\includegraphics[width=0.2\textheight]{频道选择,损失,tensorboard}}}
		\hspace{1em}
		\subfigure{\label{fig:vaeinput4}}\addtocounter{subfigure}{-2}
		\subfigure{\subfigure[输入的图片4]{\includegraphics[width=0.2\textheight]{频道选择,损失,tensorboard}}}
	\end{minipage}
	\vspace{0.2em}
\caption{输入图片对比}\label{fig:vaemnist}
\end{figure}

\chapter{公式}
勾股定理公式$a^2$如公式(\ref{eq:gougu})。

\begin{equation}\label{eq:gougu}
	a^2 + b^2 = c^2
\end{equation}

\begin{equation}\label{eq:gougu2}
	\begin{aligned}
		a^2 + b^2 &= c^2\\
		c^2 &= 1
	\end{aligned}
\end{equation}

\chapter{表格}

\begin{table}[htbp]
	\caption{一个表格}\label{tab:一个表格}
	\vspace{0.5em}\centering\wuhao
	\begin{tabular}{ccccc}
		\toprule[1.5pt]
		$D$(in) & $P_u$(lbs) & $u_u$(in) & $\beta$ & $G_f$(psi.in)\\
		\midrule[1pt]
		5 & 269.8 & 0.000674 & 1.79 & 0.04089\\
		10 & 421.0 & 0.001035 & 3.59 & 0.04089\\
		20 & 640.2 & 0.001565 & 7.18 & 0.04089\\
		\bottomrule[1.5pt]
	\end{tabular}
\end{table}



\section{本章小结}