\chapter{绪论}
\section{课题背景及研究的目的和意义}
卫星通信有通信范围广,通信服务体验好的优点,移动通信发展较为完善,用户群体巨大,且其应用服务模式十分多样,空基平台运行维护成本低,部署灵活,三者结合可以取长补短,构建一个拥有巨大优势的全球无缝覆盖的陆海空天一体化综合通信网。地面移动网络覆盖不到的地方,可由空基平台和卫星进行网络的延伸,卫星可为各偏远地区和移动载体提供连续网络连接服务,空基平台广播和多播能力可以作为卫星系统补充,为网络边缘用户提供高效数据分发服务。

近年来卫星通信和地面移动通信发展迅速,通信卫星及遥感卫星空间段频谱资源日益紧缺,地面移动通信使用频段也在向更高频段占用,空天地通信系统频谱出现交叠,为此需要研究新型异构系统网络模型及动态频谱资源分配技术,形成一个新型精细智能的动态频谱共享机制。

目前,频谱资源是政府以静态分配方式进行授权。随着技术与需求不断发展,不同的无线通信系统增加,频谱资源需求扩张。同时,有些实际已分配频谱在某些时段和地理未知上并未被充分利用,导致了频谱资源利用效率较低。处于卫星通信L波段(1GHz至2GHz)和C波段(4GHz至8GHz)同5G系统采用的3GHz至5GHz出现重叠,且日后随着移动通信要求的全频段接入能力,无线通信系统将存在大量频谱交叠。频谱资源是大数据时代最重要的战略性储备之一,频谱资源利用效率低下将限制相关技术和产业的发展。故推行异构系统的频谱共享将为新一代信息通信技术应用提供“大带宽,极高速,更广阔,超便捷”的基础资源保障。以此提升相关企业和产业的竞争优势,提高其科技化水平和总体经济效益,提高相关产业的核心竞争力。对于系统存在异构,超大带宽 ,多用户竞争并存等特点,现有频谱共享技术并不适用,且目前对于星地融合方面研究尚浅,大多停留在假想与猜测层面,频谱共享技术还有诸多问题需要解决,因此本课题具有重大的理论研究意义和实际应用价值。

\section{异构系统频谱分配的发展概况}
\subsection{异构系统的发展现状}

空天地一体化近年来研究火热,国际电信联盟(ITU)等机构对空间段以及地面段通信系统的无线电频谱使用进行了规范,认知无线电,干扰抑制等相关技术的提出为频谱资源分配提供新的解决途径。
Liu提出SAGIN概念,并对近些年相关工作进行了研究\cite{8368236}。其指出现存空天地一体化网络主要有美国国防部GIG系统,NASA军用系统,O3b系统,Iridim全球语音系统,Globalstar蜂窝电话系统,但其均未成为完整的空天地一体化多类型服务系统。
Huang等人对于跨网络结构设计进行了研究,提出基于TCP/IP模型的网络设计,用以实现以QoS为评价标准的用户体验优化\cite{6587995}。跨层设计允许在所有五个层之间共享信息,以改善无线网络功能,包括安全性,QoS和移动性。通过两种方式对跨层设计进行分类。一方面,通过如何在五层之间共享信息,跨层设计可以分为两类:非管理者方法和管理者方法。另一方面,通过网络的组织,跨层设计可以分为两类:集中式方法和分布式方法。总结了跨层设计的挑战,包括共存,信令,缺少通用跨层设计以及分层体系结构的破坏。
目前已有学者对于空天地一体化网络构架物理物理层面的具体问题进行研究\cite{7879675}。由于卫星通信系统中卫星转发器与地面终端之间的距离,多径衰落和阴影效应严重影响了卫星通信。 这些因素严重影响通信质量和系统频谱效率。他们指出由于信道时频空的三维高度变化性,长距节点高强度移动性会造成高误码率以及频带不对称,间歇中断等问题,同时也提出了一些物理层面既有建设性意义的想法,研究为实际信道仿真的实施提供了重要的意义,可广泛应用于卫星通信系统的研究。。
对于一体化网络层,Zhou等人研究了启发式延时策略解决移交丢包问题,适用于时延容限网络之中,降低丢包率\cite{8116396}。
此外Gangfei提出混合IP网络,可以降低交换时间,提高交换成功率\cite{7925019}。


\subsection{频谱分配国内外研究现状}

2001年,英国萨里大学保罗等人首次提出动态频谱分配概念。自2002年始,美国联邦通信委员会(FCC),电子与电气工程师协会,欧盟第七框架计划(FP7)都开始利用认知无线电对相关资源进行分配。后续频谱感知技术及共享频谱池技术也愈加完善。2010年,美国出台“频谱高速公路计划”,利用新型系统框架和频谱结构进行通信扩容。欧盟也成立了METIS2020项目组,推进频谱共享技术的研究。我国多个“863”计划,“973”计划等也在同时积极跟进。

对于单一网络,廖晓闽等人总结了传统蜂窝网络资源分配问题\cite{廖晓闽2019基于深度强化学习的蜂窝网资源分配算法}。其大多使用传统的博弈论理论\cite{6998030}。此论文研究了蜂窝网络下的小区间D2D通信的资源分配问题,其中D2D链接位于两个相邻小区的重叠区域中。提出了三种关于资源分配问题的小区间D2D方案,开发了重复的游戏模型,将每个参与者的效用定义为使用无线电资源从蜂窝和D2D通信中获得的收益。提出了一种基于均衡推导的资源分配算法和协议。数值结果表明,所开发的模型不仅显着提高了系统的性能,包括求和率和求和率增益,而且还揭示了小区间D2D场景的资源配置。
对于图论着色理论\cite{7833211},在全双工蜂窝网络中采用了D2D通信概念。在这样的场景中,通过重用蜂窝上行链路和下行链路的频谱资源,允许用户设备(UE)彼此通信。研究联合资源块分配和发射功率分配问题,以优化网络性能和频谱利用率。用图论对所研究的场景进行建模,并提出一种基于图着色的资源共享方案,以可接受的复杂度有效地解决联合优化问题,方案的性能通过蒙特卡洛仿真评估。
利用遗传算法\cite{8336853}的文章,提出了一种基于遗传算法的方法来最小化干扰并最大化频谱效率。 遗传算法的优点之一是,它可以通过局部搜索搜索空间的不同部分而脱离局部最大值,并向全局最大值演化。 由于D2D底层蜂窝网络会降低信号干扰加噪声比(SINR),因此,对于蜂窝用户,应考虑使用最小SINR。 数值评估表明,该技术在频谱效率和干扰减轻方面具有出色的性能。

相关研究人员使用传统算法基于导航数据的空天地网络频谱分配,此文创新性的使用空基无人机平台,对数据传输和资源卸载进行优化,提出了一种导航数据辅助的最优机会频谱接入方案,用于异构UAV网络中的无线通信,以通过灵活地调度频谱子带来实现最大化的数据速率。  此外,频谱分配过程被表述为优化问题。 仿真结果也被提出来证明与现有方案相比,该方案具有显着的性能改进。可惜对多样环境适应性较差\cite{7127619}。
Kato给出了使用人工智能解决空天地一体化问题面临的诸多挑战,比如中心控制结构简单但时延较大,分布式的多样兼容问题,强移动性和异构网络整合问题以及功率合理控制问题等\cite{8612450}。众所周知,由于缺乏网络资源和有限的覆盖范围,传统的地面通信技术的发展不能为所有用户提供公平,高质量的服务。为了补充地面连接,特别是对于农村,受灾地区或其他难以服务区域的用户,已经利用卫星,无人机和气球来中继通信信号。在此基础上,提出了SAGIN,以改善用户的QoE。但是,与现有网络(例如ad hoc网络和蜂窝网络)相比,SAGIN由于三个网段的各种特性而变得更加复杂。为了提高SAGIN的性能,研究人员面临许多前所未有的挑战。在此文中,作者提出了AI技术来优化SAGIN,因为AI技术已在许多应用程序中显示出其主要优势。首先分析SAGIN的几个主要挑战,并解释AI如何解决这些问题。然后,以卫星流量均衡为例,提出一种基于深度学习的方法来提高流量控制性能。仿真结果表明,深度学习技术可以作为提高SAGIN性能的有效工具。

另有学者使用深度学习对相关信道状态进行部分观测后进行训练和估计,获取最大化累积奖赏,但对于变化速率较快的情况算法不理想\cite{8303773}。考虑一个动态的多通道访问问题,其中多个相关的通道遵循未知的联合马尔可夫模型,并且用户选择通道来传输数据。目的是找到一种策略,以使成功传输的预期长期数量最大化。此文中该问题被表述为具有未知系统动力学的部分可观察到的马尔可夫决策过程。为了克服未知动力学和过高计算的挑战,应用了强化学习的概念并实现了深度\textit{Q}网络(DQN)。首先研究具有已知系统动力学的固定模式信道切换的最佳策略,并通过仿真显示DQN可以在不了解系统统计信息的情况下达到相同的最佳性能。
此外,Naparstek介绍使用深度强化学习对多用户进行冲突避免,但并未考虑信道状态的估计,算法效果一般\cite{8254101}。文章考虑了在多通道无线网络中最大化网络实用性的动态频谱访问问题。共享带宽被分为\textit{K}个正交信道。在每个时隙的开始,每个用户选择一个信道并以一定的传输概率发送一个分组。在每个时隙之后,已经发送了分组的每个用户接收指示其分组是否被成功递送的本地观察(即,ACK信号)。目的是一种用于访问频谱的多用户策略,该策略以分布式方式最大化某个网络效用,而无需用户之间的在线协调或消息交换。开发了一种基于深度多用户强化学习的新型分布式动态频谱访问算法。具体来说,在每个时隙,每个用户都基于用于使目标函数最大化的经过训练的Deep-Q网络,将其当前状态映射到频谱访问动作。开发了系统动力学的博弈论分析,以建立算法实现的设计原理。实验结果证明了该算法的强大性能。
程军使用深度学习对认知无线电网络进行功率控制,但其对于历史信息利用并不充分,致使智能体对当前状态估计不够准确\cite{Li2018Intelligent}。考虑由主要用户和次要用户组成的认知无线电系统中的频谱共享问题。为了辅助次要用户,在空间上部署了一组传感器节点,以在无线环境中的不同位置收集接收到的信号强度信息。开发了一种基于深度强化学习的方法,辅助用户可以使用该方法智能地调整其传输功率,以便在与主要用户进行几轮交互之后,两个用户都可以成功传输自己所需的数据并达到所需的服务质量。实验结果表明,次要用户可以在几个步骤内从任何初始状态有效地与主要用户互动,以达到目标状态(定义为两个用户都可以成功传输其数据的状态)。

学者Kwasinski在认知无线电的框架下提出基于用户使用体验为判断准则的资源分配算法,同时使用深度强化学习进行算法的优化,使用多代理DQN技术,多代理协同交互,被不同DQN网络控制,一体化学习过程,使用迁移学习减少新接入终端学习任务,提出了一种基于深度强化学习的认知无线电底层动态频谱访问(DSA)技术,该技术执行分布式联合多资源分配,以满足主链路干扰约束并最大化次网络性能(通过平均意见得分( MOS)指标。 使用MOS作为性能指标,可以无缝集成不同流量的资源。 通过利用深度\textit{Q}网络(DQN)算法,先进的深度强化学习方法和神经网络来近似\textit{Q}作用值函数,解决了资源分配问题。 此外,通过将转移学习合并到学习过程中来改善学习过程。 仿真结果表明,与不使用转移学习和标准\textit{Q}学习的DQN算法相比,转移学习将收敛的迭代次数减少了大约四分之一。美中不足的是算法仅针对于数量较少的次级用户效果较好,对于复杂情况适用和处理能力稍弱\cite{8403658}。

从以上研究现状可以看出,空天地一体化方面现有研究尚浅,大多停留在假想与猜测层面,网络结构不够具体和完善,待解决问题较多。频谱共享技术考虑单个系统的情况较多,研究涉及的频段也相对较窄。对于处于1GHz至90GHz卫星通信系统以及具备全频段接入的未来地面通信系统组成的天空地异构多用户情况下的频谱共享几乎没有项目进行研究,并且各种系统的用户是具备不同的参数的异构多用户。使用传统算法的资源分配只适合于特定模型构架的设置,对于复杂环境适用性较差,反观现有机器学习相关方面研究成果则显示出机器学习高适应性的优点,适用于多变的环境,可惜已有算法稳定性和效率都稍显不足,成为技术实际应用的瓶颈。在此背景下,为了解决日益突出的异构多系统用户用频矛盾,最大限度的提高频谱利用率和用户QoS要求,研究空天地异构多系统频谱共享技术就显得尤为重要。

\section{本文主要的研究内容}

本项目通过适当的抽象与假设,将异构系统的频谱分配问题简化为新型通信框架下的统一频谱分配问题。本文给出了一种用户中心分簇框架,利用深度强化学习的感知和决策能力通过对终端,用户进行训练,达到智能化接入目的,使智能体对历史信息进行感知,对当前信道状态进行预测和判断,并进行多用户冲突避免,进行功率控制。使用深度神经网络融合长短期记忆网络(LSTM)构造前向传输网络,使用\textit{Q}学习算法损失函数进行网络评价,采取BP算法进行网络节点参数训练,达到整个通信系统的通信速率最大化。

本文章节安排如下,第二章叙述使用的深度强化学习算法相关理论,第三章对于设计的问题背景和框架进行明确,并给出相关问题的数学模型。第四章对于算法和问题结合以及仿真结果的展示和分析,并对已完成工作进行评价以及后续工作的展望。