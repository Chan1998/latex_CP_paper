\chapter{绪论}
\section{课题背景及研究的目的和意义}
卫星通信有通信范围广,通信服务体验好的优点,移动通信发展较为完善,用户群体巨大,且其应用服务模式十分多样,空基平台运行维护成本低,部署灵活,三者结合可以取长补短,构建一个拥有巨大优势的全球无缝覆盖的陆海空天一体化综合通信网。地面移动网络覆盖不到的地方,可由空基平台和卫星进行网络的延伸,卫星可为各偏远地区和移动载体提供连续网络连接服务,空基平台广播和多播能力可以作为卫星系统补充,为网络边缘用户提供高效数据分发服务。

近年来卫星通信和地面移动通信发展迅速,通信卫星及遥感卫星空间段频谱资源日益紧缺,地面移动通信使用频段也在向更高频段占用,空天地通信系统频谱出现交叠,为此需要研究新型异构系统网络模型及动态频谱资源分配技术,形成一个新型精细智能的动态频谱共享机制。

目前,频谱资源是政府以静态分配方式进行授权。随着技术与需求不断发展,不同的无线通信系统增加,频谱资源需求扩张。同时,有些实际已分配频谱在某些时段和地理未知上并未被充分利用,导致了频谱资源利用效率较低。处于卫星通信L波段(1GHz至2GHz)和C波段(4GHz至8GHz)同5G系统采用的3GHz至5GHz出现重叠,且日后随着移动通信要求的全频段接入能力,无线通信系统将存在大量频谱交叠。频谱资源是大数据时代最重要的战略性储备之一,频谱资源利用效率低下将限制相关技术和产业的发展。故推行异构系统的频谱共享将为新一代信息通信技术应用提供“大带宽,极高速,更广阔,超便捷”的基础资源保障。以此提升相关企业和产业的竞争优势,提高其科技化水平和总体经济效益,提高相关产业的核心竞争力。对于系统存在异构,超大带宽 ,多用户竞争并存等特点,现有频谱共享技术并不适用,且目前对于星地融合方面研究尚浅,大多停留在假想与猜测层面,频谱共享技术还有诸多问题需要解决,因此本课题具有重大的理论研究意义和实际应用价值。

\section{异构系统频谱分配的发展概况}
\subsection{异构系统的发展现状}

空天地一体化近年来研究火热,国际电信联盟(ITU)等机构对空间段以及地面段通信系统的无线电频谱使用进行了规范,认知无线电,干扰抑制等相关技术的提出为频谱资源分配提供新的解决途径。
Liu提出SAGIN概念,并对近些年相关工作进行了研究\cite{8368236}。其指出现存空天地一体化网络主要有美国国防部GIG系统,NASA军用系统,O3b系统,Iridim全球语音系统,Globalstar蜂窝电话系统,但其均未成为完整的空天地一体化多类型服务系统。
Huang等人对于跨网络结构设计进行了研究,提出基于TCP/IP模型的网络设计,用以实现以QoS为评价标准的用户体验优化\cite{6587995}。
目前已有学者对于空天地一体化网络构架物理物理层面的具体问题进行研究\cite{7879675}。他们指出由于信道时频空的三维高度变化性,长距节点高强度移动性会造成高误码率以及频带不对称,间歇中断等问题,同时也提出了一些物理层面既有建设性意义的想法。
对于一体化网络层,Zhou等人研究了启发式延时策略解决移交丢包问题,适用于时延容限网络之中,降低丢包率\cite{8116396}。
此外Gangfei提出混合IP网络,可以降低交换时间,提高交换成功率\cite{7925019}。


\subsection{频谱分配国内外研究现状}

2001年,英国萨里大学保罗等人首次提出动态频谱分配概念。自2002年始,美国联邦通信委员会(FCC),电子与电气工程师协会,欧盟第七框架计划(FP7)都开始利用认知无线电对相关资源进行分配。后续频谱感知技术及共享频谱池技术也愈加完善。2010年,美国出台“频谱高速公路计划”,利用新型系统框架和频谱结构进行通信扩容。欧盟也成立了METIS2020项目组,推进频谱共享技术的研究。我国多个“863”计划,“973”计划等也在同时积极跟进。

对于单一网络,廖晓闽等人总结了传统蜂窝网络资源分配问题\cite{廖晓闽2019基于深度强化学习的蜂窝网资源分配算法}。其大多使用传统的博弈论理论\cite{6998030},图论着色理论\cite{7833211},和利用遗传算法\cite{8336853}等方法进行分配,本文主要讨论使用机器学习的新方法和理论,对产传统算法不再详细介绍。
相关研究人员使用传统算法基于导航数据的空天地网络频谱分配,此文创新性的使用空基无人机平台,对数据传输和资源卸载进行优化,可惜对多样环境适应性较差\cite{7127619}。
Kato给出了使用人工智能解决空天地一体化问题面临的诸多挑战,比如中心控制结构简单但时延较大,分布式的多样兼容问题,强移动性和异构网络整合问题以及功率合理控制问题等\cite{8612450}。
另有学者使用深度学习对相关信道状态进行部分观测后进行训练和估计,获取最大化累积奖赏,但对于变化速率较快的情况算法不理想\cite{8303773}。
此外,Naparstek介绍使用深度强化学习对多用户进行冲突避免,但并未考虑信道状态的估计,算法效果一般\cite{8254101}。
程军使用深度学习对认知无线电网络进行功率控制,但其对于历史信息利用并不充分,致使智能体对当前状态估计不够准确\cite{Li2018Intelligent}。
学者Kwasinski在认知无线电的框架下提出基于用户使用体验为判断准则的资源分配算法,同时使用深度强化学习进行算法的优化,使用多代理DQN技术,多代理协同交互,被不同DQN网络控制,一体化学习过程,使用迁移学习减少新接入终端学习任务,美中不足的是算法仅针对于数量较少的次级用户效果较好,对于复杂情况适用和处理能力稍弱\cite{8403658}。

从以上研究现状可以看出,空天地一体化方面现有研究尚浅,大多停留在假想与猜测层面,网络结构不够具体和完善,待解决问题较多。频谱共享技术考虑单个系统的情况较多,研究涉及的频段也相对较窄。对于处于1GHz至90GHz卫星通信系统以及具备全频段接入的未来地面通信系统组成的天空地异构多用户情况下的频谱共享几乎没有项目进行研究,并且各种系统的用户是具备不同的参数的异构多用户。使用传统算法的资源分配只适合于特定模型构架的设置,对于复杂环境适用性较差,反观现有机器学习相关方面研究成果则显示出机器学习高适应性的优点,适用于多变的环境,可惜已有算法稳定性和效率都稍显不足,成为技术实际应用的瓶颈。在此背景下,为了解决日益突出的异构多系统用户用频矛盾,最大限度的提高频谱利用率和用户QoS要求,研究空天地异构多系统频谱共享技术就显得尤为重要。

\section{本文主要的研究内容}

本项目通过适当的抽象与假设,将异构系统的频谱分配问题简化为新型通信框架下的统一频谱分配问题。本文给出了一种用户中心分簇框架,利用深度强化学习的感知和决策能力通过对终端,用户进行训练,达到智能化接入目的,使智能体对历史信息进行感知,对当前信道状态进行预测和判断,并进行多用户冲突避免,进行功率控制。使用深度神经网络融合长短期记忆网络(LSTM)构造前向传输网络,使用\textit{Q}学习算法损失函数进行网络评价,采取BP算法进行网络节点参数训练,达到整个通信系统的通信速率最大化。

本文章节安排如下,第二章叙述使用的深度强化学习算法相关理论,第三章对于设计的问题背景和框架进行明确,并给出相关问题的数学模型。第四章对于算法和问题结合以及仿真结果的展示和分析,并对已完成工作进行评价以及后续工作的展望。