\begin{conclusions}

本文在空天地一体化通信框架下通过使用深度强化学习解决异构系统频谱资源分配相关问题,在簇头分配,信道选择,功率控制三方面取得令人满意效果。为了适应空天地一体化通信构架,本文建立了以用户为中心,动态调整大小的分簇模型。摆脱蜂窝网小区制度的限制,将簇头和簇内用户紧密联系,满足未来通信高密度,高质量,高速度的要求。并基于创建的分簇模型提出了完整的通信链路建立流程,将整个流程具体细化为簇头分配,信道选择,功率控制三方面并对相关问题进行数学建模的建立,实现分布式资源分配,提高用户信息传输体验。传统算法针对本文问题性能不足,为此本文利用DRQN网络解决异构网络通信链路建立问题,并依据具体问题对已有DRQN算法进行了改进和增强,取得不错成效。对于深度强化学习在通信相关问题上应用,进行了初步尝试,验证了其思路方法可行性,并得出了可解释性较强的仿真结果,并通过与传统算法的性能对比证明了机器学习在相关问题解决方面的独特优势。

虽然DRQN这种深度强化学习在应对频谱分配问题上拥有不错的效果,但通过认真的分析和反思,我就发现的一些问题和不足进行阐述,并给出后续工作的方向展望。首先,对于异构系统分簇构架下的频谱分配问题,虽然我们在地面段创新性的采取了分布式分簇模型,很好的解决了地面段频谱资源的分配。但对于卫星和空中无人机平台的研究还需要进一步跟进,比如卫星间协同工作问题以及无人机资源卸载问题,同时在物理层面高频段电波信号的损耗问题以及网络层不同系统协议同一问题等需要进一步研究完善。另外对于异构用户的评价体系也为给出详细具体的实施方案,仅仅使用QoS是远远不够的,近年来有学者研究的使用QoE作为评价指标可能是解决方案。至于资源分配算法层面,我们使用的DRQN算法虽然让DQN网络具有时序预测能力,但DQN方法本身固有缺陷并未克服,比如训练不稳定,易得到局部最优解。对于我们的项目,这里也有需要后续改进地方,比如虽然在解决簇头选择,信道分配,功率控制时使用的都是同一算法与同一网络框架,但由于三个问题具体细节的不同使得三个网络不能共享网络参数,实现同步训练,后续将其统一与结合将是待解决问题。最后是文章使用的算法是基于价值的深度强化学习模式,对于输出动作维度限制较大,后续工作可以考虑将基于策略的DDPG,A3C等算法进行考虑,或许会得到更好的结果。

\end{conclusions}
