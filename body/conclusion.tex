\begin{conclusions}

本文在空天地一体化通信框架下通过使用深度强化学习解决异构系统频谱资源分配相关问题,在簇头分配,信道选择,功率控制三方面取得令人满意效果。本文创新点如下:
(1)创新性的建立了以用户为中心,动态调整大小的分簇模型。首次摆脱蜂窝网小区制度的限制,将簇头和簇内用户紧密联系,满足未来通信高密度,高质量,高速度的要求。
(2)基于创建分簇模型提出了完整的通信链路建立流程,将整个流程具体细化为簇头分配,信道选择,功率控制三方面并对相关问题进行数学建模的建立,实现分布式资源分配,提高用户信息传输体验。
(3)创新性的利用DRQN网络解决异构网络通信链路建立问题,并取得不错成效。
对于深度强化学习在通信相关问题上应用,进行了初步尝试,验证了其思路方法可行性,并得出了可解释性较强的仿真结果,并通过与传统算法的性能对比证明了机器学习在相关问题解决方面的独特优势。

但由于机器学习方法本身固有问题,对异构系统频谱分配问题仍有改进和提升空间。首先是将本文提出的三个问题如何使用同一网络解决问题,如果能够实现一体化训练将会对框架整体带来巨大增益。其次是任务奖赏设置稍显单一,对于异构系统需要更多样更全面的评价体系。最后是文章使用的算法是基于价值的深度强化学习模式,对于输出动作维度限制较大,后续工作可以考虑将基于策略的DDPG,A3C等算法进行考虑,或许会得到更好的结果。
\end{conclusions}
